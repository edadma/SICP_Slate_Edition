\section*{Programming in Slate}
\addcontentsline{toc}{section}{Programming in Slate}

We need an appropriate language for describing processes, and we will use for this purpose the programming language Slate. Just as our everyday thoughts are usually expressed in our natural language (such as English, French, or Japanese), and descriptions of quantitative phenomena are expressed with mathematical notations, our procedural thoughts will be expressed in Slate.

Slate is a dynamically-typed, expression-oriented programming language designed for both embedded systems and general computing. It features significant whitespace (like Python), prototype-based objects (like JavaScript), and an expression-oriented syntax where control flow constructs return values. The language uses reference counting for memory management, making it suitable for systems programming while maintaining the expressive power needed for exploring computational concepts.

Despite its modern design, Slate provides all the tools we need for studying fundamental programming concepts. The language supports first-class functions, closures, and powerful abstraction mechanisms. It includes features like pattern matching and destructuring that will enhance our exploration of data structures and algorithms. Most importantly for our study, Slate treats procedures as first-class objects, allowing us to manipulate programs as data---a capability that will prove essential in later chapters.

If Slate is not yet a mainstream language, why are we using it as the framework for our discussion of programming? Because the language possesses unique features that make it an excellent medium for studying important programming constructs and data structures. The most significant of these features is the fact that Slate descriptions of processes, called procedures, can themselves be represented and manipulated as Slate data. The importance of this is that there are powerful program-design techniques that rely on the ability to blur the traditional distinction between ``passive'' data and ``active'' processes. As we shall discover, Slate's flexibility in handling procedures as data makes it one of the most convenient languages in existence for exploring these techniques. The ability to represent procedures as data also makes Slate an excellent language for writing programs that data, such as the interpreters and compilers that support computer languages. Above and beyond these considerations, programming in Slate is great fun.